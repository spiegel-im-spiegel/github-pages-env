\documentclass{jlreq}
\usepackage[jis2004,deluxe]{luatexja-preset} % Japanese fonts
\setmonojfont[AutoFakeSlant,BoldItalicFeatures={FakeSlant},Scale=MatchLowercase]{HackGenNerd} % use HackGenNerd
\setmonofont[AutoFakeSlant,BoldItalicFeatures={FakeSlant},Scale=MatchLowercase]{HackGenNerd} % use HackGenNerd

\usepackage{graphicx,xcolor}
% \usepackage{listings}
\usepackage{listings-golang} % import this package after listings
\lstset{
	frame=single,
	basicstyle=\small\ttfamily,
	tabsize=4,
	commentstyle=\color{darkgray},
	keywordstyle=\color{brown}\bfseries,
	stringstyle=\color{blue},
	showstringspaces=false
}

\begin{document}

\section{Go 言語による Hello World}

\begin{lstlisting}[language=Golang]
package main

import "fmt"

func main() {
	for i := 0; i < 10; i++ {
		fmt.Println("Hello, world") //Hello, 世界
	}
}
\end{lstlisting}

\section{シェルスクリプト}

\begin{lstlisting}[language=sh]
#!/bin/sh
for i in `seq 100`; do
  j="$i"
  if [ `expr $i % 3` == 0 ]; then echo -n 'Fizz'; j=''; fi
  if [ `expr $i % 5` == 0 ]; then echo -n 'Buzz'; j=''; fi
  echo "$j"
done
\end{lstlisting}

\end{document}
