\documentclass[fleqn]{ltjsarticle}
\usepackage[no-math,sourcehan]{luatexja-preset} % Japanese fonts

\usepackage{amsmath} % amsmath packages
\usepackage{unicode-math} % using unicode/OpenType maths fonts
\unimathsetup{%
    math-style=TeX,%
    bold-style=TeX,%
    sans-style=italic%
}

% Import text fonts
%\setmainfont[Scale=MatchLowercase]{TeXGyreTermes}
\setmainfont[Scale=MatchLowercase]{STIX Two Text}
\setsansfont[Scale=MatchLowercase]{TeXGyreHeros}
\setmonofont[AutoFakeSlant,BoldItalicFeatures={FakeSlant},Scale=MatchLowercase]{Inconsolatazi4}

% Import math fonts
%\setmathfont{TeXGyrePagella-Math}
%\setmathfont{Neo Euler}
\setmathfont{STIX Two Math}


\begin{document}

\section*{数式表現}

\subsection*{インライン数式}

エネルギーと質量には $E=mc^2$ の関係がある。

\subsection*{別行立て数式}

エネルギーと質量には \[E=mc^2\] の関係がある。

\subsection*{ちょっと複雑な式も書いてみる}

``?`!`But aren't Kafka's \emph{Schlo{\ss}} and {\AE}sop's {\OE}uvres
often na\"{\i}ve vis-\`a-vis the d{\ae}monic ph{\oe}nix's
official r\^ole in fluffy souffl\'es!?''

\[
  \left( \int_0^\infty \frac{\sin x}{\sqrt x} dx \right)^2 =
  \sum_{k=0}^\infty \frac{(2k)!}{2^{2k}(k!)^2} \frac{1}{2k+1} =
  \prod_{k=1}^\infty \frac{4k^2}{4k^2-1} =
  \frac{\pi}{2}
\]

\subsection*{連分数(amsmathパッケージ)}
\begin{equation}
	b_0 + \cfrac{c_1}{b_1 +
	      \cfrac{c_2}{b_2 +
		  \cfrac{c_3}{b_3 +
		  \cfrac{c_4}{b_4 + \cdots}}}}
\end{equation}

\subsection*{黒板文字(amssybパッケージ)}

$\mathbb{ABCDEFGHIJELMN}$

\end{document}
